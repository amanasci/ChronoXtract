\begin{multicols}{2}
\tableofcontents
\section{Introduction}
Time series or sequential data are ubiquitous in modern applications, covering finance, healthcare, climate science, astrophysics, particle physics and the Internet of Things (IoT). Extracting meaningful features from such data is critical for anomaly detection, forecasting, and classification tasks. Such meaningful features are hard to identify that can summarise the long-term and short-term dynamics contained in the time series/ sequential data. \\
Statistical feature extraction serves as a foundational step in transforming raw time series data into structured, interpretable representations that reveal underlying patterns, trends, and anomalies. These features act as concise numerical summaries of complex temporal dynamics, enabling downstream tasks such as classification, forecasting, anomaly detection, and model training. For instance, basic statistical measures—such as the mean , median , and mode —capture central tendencies and distributional properties, while metrics like fractional variability (the ratio of the range to the mean) quantify volatility or irregularity in the data. Such metrics are indispensable for identifying outliers, assessing stability, or comparing datasets across domains. Beyond descriptive statistics, spectral features derived from methods like Fourier analysis provide insights into periodicity and cyclical behavior, which are critical for applications such as signal processing, climate modeling, and sensor data analysis. Fourier coefficients, for example, decompose time series into constituent frequencies, enabling the detection of hidden periodic patterns or noise components. Together, these statistical and spectral features form a versatile toolkit for distilling actionable insights from raw data, bridging the gap between raw observations and higher-level analytical tasks. In the next section, we describe the details about the statistical features included in the \textit{ChronoXtract} library. 
\end{multicols}