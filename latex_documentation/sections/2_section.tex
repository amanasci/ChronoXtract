{\color{gray}\hrule}
\begin{center}
\section{Statistical Features}
% \textbf{You can add small descriptions of what the current sections describe}
\bigskip
\end{center}
{\color{gray}\hrule}
Here, we discuss the list of statistical features extracted by the ChronoXtract package. 
% \begin{multicols}{2}
\subsection{Mean}
The mean (or average) represents the central tendency of the time series data. It provides a single value summarizing the overall magnitude of the dataset. Mean is given by: 
\begin{equation}
    \mu = \frac{1}{N} \sum_{i=1}^{N} x_i
    \label{eq:mean}
\end{equation}

Algorithm:
\begin{enumerate}
    \item Sum all elements in the time series vector.
    \item Divide the sum by the length of the vector.
\end{enumerate}


\subsection{Median}
The median is the middle value of a sorted dataset. It is robust to outliers and provides a measure of central tendency that is less sensitive to extreme values than the mean.
\begin{equation}
   \text{Median} =
   \begin{cases} 
   \dfrac{n+1}{2}, & \text{if } n \text{ is odd} \\ 
   \dfrac{x_{\frac{n}{2}} + x_{\frac{n}{2} + 1}}{2}, & \text{if } n \text{ is even}
   \end{cases}
   \label{eq:median_expression}
\end{equation}

Algorithm:
\begin{enumerate}
    \item Sort the vector.
    \item Compute the median based on whether the length of the vector is odd or even.
\end{enumerate}

\subsection{Mode}
The mode is the most frequently occurring value in the dataset. It is useful for identifying dominant patterns or trends in categorical or discrete numerical data.

\begin{equation}
    
\end{equation}


% \end{multicols}