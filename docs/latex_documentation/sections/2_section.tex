{\color{gray}\hrule}
\begin{center}
\section{Statistical Features}
% \textbf{You can add small descriptions of what the current sections describe}
\bigskip
\end{center}
{\color{gray}\hrule}
Here, we discuss the list of statistical features extracted by the ChronoXtract package. 
% \begin{multicols}{2}
\subsection{Mean}
The mean (or average) represents the central tendency of the time series data. It provides a single value summarizing the overall magnitude of the dataset. Mean is given by: 
\begin{equation}
    \mu = \frac{1}{N} \sum_{i=1}^{N} x_i
    \label{eq:mean}
\end{equation}

Algorithm:
\begin{enumerate}
    \item Sum all elements in the time series vector.
    \item Divide the sum by the length of the vector.
\end{enumerate}


\subsection{Median}
The median is the middle value of a sorted dataset. It is robust to outliers and provides a measure of central tendency that is less sensitive to extreme values than the mean.
\begin{equation}
   \text{Median} =
   \begin{cases} 
   \dfrac{n+1}{2}, & \text{if } n \text{ is odd} \\ 
   \dfrac{x_{\frac{n}{2}} + x_{\frac{n}{2} + 1}}{2}, & \text{if } n \text{ is even}
   \end{cases}
   \label{eq:median_expression}
\end{equation}

Algorithm:
\begin{enumerate}
    \item Sort the vector.
    \item Compute the median based on whether the length of the vector is odd or even.
\end{enumerate}

\subsection{Mode}
The mode is the most frequently occurring value in the dataset. It is useful for identifying dominant patterns or trends in categorical or discrete numerical data.
\begin{equation}
    \text{Mode} = \text{arg} \text{max} \{ \text{count}(x_i) \}
\end{equation}

Algorithm:
\begin{enumerate}
    \item Use a HashMap to count occurrences of each value in the time series. 
    \item Iterate through the HashMap to find the key with the maximum count.
\end{enumerate}

\subsection{Variance}
Variance measures the spread or dispersion of the data around the mean. A higher variance indicates greater variability in the dataset.

\begin{equation}
    \sigma^2 = \dfrac{1}{N} \sum_{i=1}^{N} (x_i - \mu)^2
    \label{eq:variance}
\end{equation}

here $\mu$ is the mean of the dataset.

Algorithm:
\begin{enumerate}
    \item Compute the mean of the time series.
    \item Calculate the squared difference between each element and the mean.
    \item Sum the squared differences and divide by the length of the vector.
\end{enumerate}

\subsection{Standard Deviation}
Standard deviation is the square root of the variance. It provides a measure of the dispersion of the data around the mean. A higher standard deviation indicates greater variability in the dataset.

\begin{equation}
    \sigma = \sqrt{\sigma^2}
    \label{eq:std_dev}
\end{equation}

Algorithm:
\begin{enumerate}
    \item Compute the variance of the time series.
    \item Take the square root of the variance.
\end{enumerate}

\subsection{Skewness}
Skewness measures the asymmetry of the dataset. A positive skewness indicates a longer tail on the right side of the distribution, while a negative skewness indicates a longer tail on the left side.

\begin{equation}
    \text{Skewness} = \dfrac{\dfrac{1}{N} \sum_{i=1}^{N} (x_i - \mu)^3}{\sigma^3}
    \label{eq:skewness}
\end{equation}

Algorithm:
\begin{enumerate}
    \item Compute the mean and standard deviation of the time series.
    \item Calculate the skewness using the formula above.
\end{enumerate}

\subsection{Kurtosis}
Kurtosis measures the tailedness of the dataset. A higher kurtosis indicates heavier tails in the distribution, while a lower kurtosis indicates lighter tails.

\begin{equation}
    \text{Kurtosis} = \dfrac{\dfrac{1}{N} \sum_{i=1}^{N} (x_i - \mu)^4}{\sigma^4}
    \label{eq:kurtosis}
\end{equation}

Algorithm:
\begin{enumerate}
    \item Compute the mean and standard deviation of the time series.
    \item Calculate the kurtosis using the formula above.
\end{enumerate}

\subsection{Min Max Range}
The min-max range is the difference between the maximum and minimum values in the dataset. It provides a measure of the spread of the data.

\begin{equation}
    \text{min-max range} = \text{max}(x) - \text{min}(x)
    \label{eq:min_max_range}
\end{equation}

Algorithm:
\begin{enumerate}
    \item Find the maximum and minimum values in the time series.
    \item Calculate the difference between the maximum and minimum values.
\end{enumerate}

\subsection{Quantiles}
Quantiles divide the dataset into equal parts. The median is the 50th percentile, while the quartiles divide the dataset into four equal parts. Quantiles provide a measure of the spread and distribution of the data.

Algorithm:
\begin{enumerate}
    \item Sort the time series.
    \item Calculate the quantiles based on the desired percentage.
\end{enumerate}

\subsection{Sum}
The sum of the dataset provides a measure of the total magnitude of the data.

\begin{equation}
    \text{Sum} = \sum_{i=1}^{N} x_i
    \label{eq:sum}
\end{equation}

Algorithm:
\begin{enumerate}
    \item Sum all elements in the time series vector.
\end{enumerate}

\subsection{Absolute Energy}
Absolute energy is the sum of the squared values in the dataset. It provides a measure of the total energy in the data.

\begin{equation}
    \text{Absolute Energy} = \sum_{i=1}^{N} x_i^2
    \label{eq:absolute_energy}
\end{equation}

Algorithm:
\begin{enumerate}
    \item Square each element in the time series.
    \item Sum the squared values.
\end{enumerate}





% \end{multicols}