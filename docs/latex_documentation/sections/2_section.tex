{\color{gray}\hrule}
\begin{center}
\section{Statistical Features}
% \textbf{You can add small descriptions of what the current sections describe}
\bigskip
\end{center}
{\color{gray}\hrule}
Here, we discuss the list of statistical features extracted by the ChronoXtract package. 
% \begin{multicols}{2}
\subsection{Mean}
The mean (or average) represents the central tendency of the time series data. It provides a single value summarizing the overall magnitude of the dataset. Mean is given by: 
\begin{equation}
    \mu = \frac{1}{N} \sum_{i=1}^{N} x_i
    \label{eq:mean}
\end{equation}

Algorithm:
\begin{enumerate}
    \item Sum all elements in the time series vector.
    \item Divide the sum by the length of the vector.
\end{enumerate}


\subsection{Median}
The median is the middle value of a sorted dataset. It is robust to outliers and provides a measure of central tendency that is less sensitive to extreme values than the mean.
\begin{equation}
   \text{Median} =
   \begin{cases} 
   \dfrac{n+1}{2}, & \text{if } n \text{ is odd} \\ 
   \dfrac{x_{\frac{n}{2}} + x_{\frac{n}{2} + 1}}{2}, & \text{if } n \text{ is even}
   \end{cases}
   \label{eq:median_expression}
\end{equation}

Algorithm:
\begin{enumerate}
    \item Sort the vector.
    \item Compute the median based on whether the length of the vector is odd or even.
\end{enumerate}

\subsection{Mode}
The mode is the most frequently occurring value in the dataset. It is useful for identifying dominant patterns or trends in categorical or discrete numerical data.
\begin{equation}
    \text{Mode} = \text{arg} \text{max} \{ \text{count}(x_i) \}
\end{equation}

Algorithm:
\begin{enumerate}
    \item Use a HashMap to count occurrences of each value in the time series. 
    \item Iterate through the HashMap to find the key with the maximum count.
\end{enumerate}

\subsection{Variance}
Variance measures the spread or dispersion of the data around the mean. A higher variance indicates greater variability in the dataset.

\begin{equation}
    \sigma^2 = \dfrac{1}{N} \sum_{i=1}^{N} (x_i - \mu)^2
    \label{eq:variance}
\end{equation}

here $\mu$ is the mean of the dataset.

Algorithm:
\begin{enumerate}
    \item Compute the mean of the time series.
    \item Calculate the squared difference between each element and the mean.
    \item Sum the squared differences and divide by the length of the vector.
\end{enumerate}

\subsection{Standard Deviation}
Standard deviation is the square root of the variance. It provides a measure of the dispersion of the data around the mean. A higher standard deviation indicates greater variability in the dataset.

\begin{equation}
    \sigma = \sqrt{\sigma^2}
    \label{eq:std_dev}
\end{equation}

Algorithm:
\begin{enumerate}
    \item Compute the variance of the time series.
    \item Take the square root of the variance.
\end{enumerate}

\subsection{Skewness}
Skewness measures the asymmetry of the dataset. A positive skewness indicates a longer tail on the right side of the distribution, while a negative skewness indicates a longer tail on the left side.

\begin{equation}
    \text{Skewness} = \dfrac{\dfrac{1}{N} \sum_{i=1}^{N} (x_i - \mu)^3}{\sigma^3}
    \label{eq:skewness}
\end{equation}

Algorithm:
\begin{enumerate}
    \item Compute the mean and standard deviation of the time series.
    \item Calculate the skewness using the formula above.
\end{enumerate}

\subsection{Kurtosis}
Kurtosis measures the tailedness of the dataset. A higher kurtosis indicates heavier tails in the distribution, while a lower kurtosis indicates lighter tails.

\begin{equation}
    \text{Kurtosis} = \dfrac{\dfrac{1}{N} \sum_{i=1}^{N} (x_i - \mu)^4}{\sigma^4}
    \label{eq:kurtosis}
\end{equation}

Algorithm:
\begin{enumerate}
    \item Compute the mean and standard deviation of the time series.
    \item Calculate the kurtosis using the formula above.
\end{enumerate}

\subsection{Min Max Range}
The min-max range is the difference between the maximum and minimum values in the dataset. It provides a measure of the spread of the data.

\begin{equation}
    \text{min-max range} = \text{max}(x) - \text{min}(x)
    \label{eq:min_max_range}
\end{equation}

Algorithm:
\begin{enumerate}
    \item Find the maximum and minimum values in the time series.
    \item Calculate the difference between the maximum and minimum values.
\end{enumerate}

\subsection{Quantiles}
Quantiles divide the dataset into equal parts. The median is the 50th percentile, while the quartiles divide the dataset into four equal parts. Quantiles provide a measure of the spread and distribution of the data.

Algorithm:
\begin{enumerate}
    \item Sort the time series.
    \item Calculate the quantiles based on the desired percentage.
\end{enumerate}

\subsection{Sum}
The sum of the dataset provides a measure of the total magnitude of the data.

\begin{equation}
    \text{Sum} = \sum_{i=1}^{N} x_i
    \label{eq:sum}
\end{equation}

Algorithm:
\begin{enumerate}
    \item Sum all elements in the time series vector.
\end{enumerate}

\subsection{Absolute Energy}
Absolute energy is the sum of the squared values in the dataset. It provides a measure of the total energy in the data.

\begin{equation}
    \text{Absolute Energy} = \sum_{i=1}^{N} x_i^2
    \label{eq:absolute_energy}
\end{equation}

Algorithm:
\begin{enumerate}
    \item Square each element in the time series.
    \item Sum the squared values.
\end{enumerate}

\subsection{Rolling Mean}
The rolling mean (or moving average) is the average of a specified number of previous values in the dataset. It provides a smoothed version of the time series data.
\begin{equation}
    \text{Rolling Mean} = \dfrac{1}{N} \sum_{i=t-N+1}^{t} x_i
    \label{eq:rolling_mean}
\end{equation}

Algorithm:
\begin{enumerate}
    \item Initialize a window of size N.
    \item For each time step t, calculate the mean of the previous N values.
    \item Slide the window one step forward and repeat.
    \item Continue until the end of the time series is reached.
    \item Return the rolling mean values.
\end{enumerate}

\subsection{Rolling Variance}
The rolling variance is the variance of a specified number of previous values in the dataset. It provides a measure of the variability of the data over time.
\begin{equation}
    \text{Rolling Variance} = \dfrac{1}{N} \sum_{i=t-N+1}^{t} (x_i - \mu)^2
    \label{eq:rolling_variance}
\end{equation}
Algorithm:
\begin{enumerate}
    \item Initialize a window of size N.
    \item For each time step t, calculate the variance of the previous N values.
    \item Slide the window one step forward and repeat.
    \item Continue until the end of the time series is reached.
    \item Return the rolling variance values.
\end{enumerate}

\subsection{Expanding Sum}
The expanding sum is the cumulative sum of the dataset up to the current time step. It provides a measure of the total magnitude of the data over time.
\begin{equation}
    \text{Expanding Sum} = \sum_{i=1}^{t} x_i
    \label{eq:expanding_sum}
\end{equation}

Algorithm:
\begin{enumerate}
    \item Initialize a variable to store the cumulative sum.
    \item For each time step t, add the current value to the cumulative sum.
    \item Continue until the end of the time series is reached.
    \item Return the expanding sum values.
\end{enumerate}

\subsection{Exponential Moving Average}
The exponential moving average (EMA) is a weighted average of the previous values in the dataset, where more recent values have a higher weight. It provides a smoothed version of the time series data that reacts more quickly to recent changes.
\begin{equation}
    \text{EMA}_t = \alpha x_t + (1 - \alpha) \text{EMA}_{t-1}
    \label{eq:ema}
\end{equation}
where $\alpha$ is the smoothing factor, typically set to $\frac{2}{N+1}$, where N is the window size.

Algorithm:
\begin{enumerate}
    \item Initialize the EMA with the first value of the time series.
    \item For each subsequent time step t, calculate the EMA using the formula above.
    \item Continue until the end of the time series is reached.
    \item Return the EMA values.
\end{enumerate}

\subsection{Sliding Window Entropy}
Sliding window entropy is a measure of the complexity or uncertainty of the dataset over a specified window size. It provides a measure of the information content in the data.
\begin{equation}
    \text{Entropy} = -\sum_{i=1}^{N} p_i \log(p_i)
    \label{eq:entropy}
\end{equation}
where $p_i$ is the probability of each value in the dataset.

Algorithm:
\begin{enumerate}
    \item Initialize an empty vector to store entropy values.
    \item Validate inputs: return empty vector if window size is 0, window size exceeds series length, or bins is 0.
    \item For each position i from 0 to (n - window):
        \begin{enumerate}
            \item Extract a slice of the series from i to i + window.
            \item Find the minimum and maximum values in the window slice.
            \item Calculate the range (max\_value - min\_value).
            \item If the range equals 0, add 0.0 to the entropy vector and continue to the next iteration.
            \item Otherwise:
                \begin{enumerate}
                    \item Initialize a histogram with 'bins' number of bins, each set to 0.
                    \item For each value in the window:
                        \begin{enumerate}
                            \item Determine which bin the value belongs to using the formula: bin = floor((value - min\_value) / range * bins).
                            \item Ensure the bin index doesn't exceed the number of bins.
                            \item Increment the count for that bin.
                        \end{enumerate}
                    \item Calculate entropy as follows:
                        \begin{enumerate}
                            \item For each bin with count $> 0$, calculate probability $p = count / window\_size$.
                            \item Sum up $(-p * \ln(p))$ for all bins with counts $> 0$.
                        \end{enumerate}
                    \item Add the calculated entropy value to the entropy vector.
                \end{enumerate}
        \end{enumerate}
    \item Return the vector of entropy values.
\end{enumerate}